\documentclass[sotsuron]{kuee}
\usepackage[dvipdfmx]{graphicx}

\graphicspath{{figures/}}

\title{自己発振昇圧回路を用いた\\低待機電力温度センサ集積回路}
\etitle{A Low Standby Power Temperature Sensor IC Using a Self-Oscillating Voltage Doubler}
\author{太田 裕真}
\eauthor{Yuma Ota}
\professor{新津 葵一 教授}
%\course{京都大学大学院情報学研究科}
%\department{通信情報システム専攻}
\date{令和8年1月10日}

%%% 本文
\begin{document}
\maketitle			% 表題を出力
\begin{eabstract}		% 英文要旨を出力
English abstract goes here.
\end{eabstract}
\tableofcontents		% 目次を出力


\chapter{使い方}
基本的な使い方は通常のLaTeXと同様.
図表はfiguresフォルダに入れる.
BibTeXを使用する場合は,main.bibに参考文献を記述する.
図表を挿入すると,図\ref{fig:dummy}のように,ページの最後にまとめて出力される.

\begin{figure}
    \centering
    \includegraphics[width=0.5\linewidth]{dummy.pdf}
    \caption{ダミー図}
    \label{fig:dummy}
\end{figure}

そのほかの細かい使い方は,sample.pdfを参照のこと.

\chapter{GitHubとの連携}
作業ブランチからmasterブランチにPull Requestを出すと,GitActionsが自動的に動作してtextlintが走る,らしい.
ここら辺の仕組みはあまりよくわかっていない.

%======================================================================
%		謝辞
%======================================================================
\begin{acknowledgements}

\end{acknowledgements}



%======================================================================
%		参考文献
%======================================================================
\bibliographystyle{kueethesis}
\bibliography{main}



%======================================================================
%		付録
%======================================================================
\appendix
\chapter{}

\chapter{}


\end{document}
% Local Variables:
% fill-column: 70
% End: